\documentclass{beamer}
\usepackage[spanish]{babel}
\usepackage[utf8]{inputenc}
\usepackage{graphicx}
\graphicspath{ {./images/}}

\usetheme{Madrid}
\usecolortheme{default}

%------------------------------------------------------------
%This block of code defines the information to appear in the
%Title page
  \title[PC3 Estadistica Aplicada] %optional
{Aplicacion de test no parametricos}

\subtitle{
  Caso: Distribucion de notas de estudiantes de la acadmia
    Trilce en los ultimos 3 años
}

\author % (optional)
{
  Alvarez \and Bautista \and Burga \and
  Casanova \and  Cuyate
}

\institute
{
  Facultad de Ingenieria Industrial y de Sistemas\\
  \textbf{Universidad Nacional de Ingenieria}
}

\date
{Diciembre 2022}

% \logo{\includegraphics[height=1cm]{overleaf-logo}}

%End of title page configuration block
%------------------------------------------------------------

%------------------------------------------------------------
%The next block of commands puts the table of contents at the
%beginning of each section and highlights the current section:

\AtBeginSection[]
{
  \begin{frame}
    \frametitle{Tabla de Contenido}
    \tableofcontents[currentsection]
  \end{frame}
}
%------------------------------------------------------------


\begin{document}

%The next statement creates the title page.
\frame{\titlepage}


%---------------------------------------------------------
%This block of code is for the table of contents after
%the title page
\begin{frame}
\frametitle{Tabla de Contenido}
\tableofcontents
\end{frame}
%---------------------------------------------------------
\section{Objetivos}

%---------------------------------------------------------
\begin{frame}

\frametitle{Objetivos del trabajo}

\begin{alertblock}{General}
  Conocer si el ciclo de repaso util para los postulantes a la
  Universidad Nacional de Ingenieria
\end{alertblock}
\end{frame}

\begin{frame}
\frametitle{Objetivos especificos}

\begin{columns}
\column{0.5\textwidth}

  \begin{itemize}
      \item Probar si alguna de las medianas de la distribución de los últimos años para
      los simulacros escolares difiere con un nivel de significancia del 5\% con la prueba de Kruskal-Wallis.
      \item Comparar el desempeño de los estudiantes antes de tomar un ciclo
        de repaso y despues de este.
      \item Comparar la evolucion del desempeño de aquellos que tomaron mas de un simulacro.

  \end{itemize}

\column{0.5\textwidth}

  \begin{itemize}
      \item Comparar la potencia entre el test de \textit{Wilcoxon} y
      el \textit{Test de signos}
      \item Los cientificos de datos poseen un mejor distribucion de ingresos
        que los ingenieros de datos
      \item El sector \textit{(publico / privado)} al que pertenece un trabajador
        es causa de la diferencia de salarios
  \end{itemize}
\end{columns}
\end{frame}

%---------------------------------------------------------

\begin{frame}
\frametitle{Hipotesis especificas}
  \begin{itemize}
      \item Las personas que trabajan una cantidad de horas superior a
        la media tienen una mejor destribucion de ingresos que aquellas
        que no lo hacen
      \item Las personas de mediana edad poseen una mejor distribucion
        de ingreso que las personas jovenes
      \item El promedio de ingresos de la poblacion mexicana es mayor
        que la peruana
  \end{itemize}

\end{frame}

%---------------------------------------------------------
\section{Metodologia}

\begin{frame}
  \frametitle{Test de Kruskal-Wallis}


\end{frame}


\begin{frame}
  \frametitle{Test de Signos}


\end{frame}

\begin{frame}
  \frametitle{Test de Wilcoxon}

  Si bien, el test de signos puede cumplir la misma funcion que el de
  \textbf{Wilcoxon}, este ultimo tiene mayor potencia al momento de
  detectar diferencia de medias.


\end{frame}

%---------------------------------------------------------


\section{Resultados y Conclusiones}

\begin{frame}
  \frametitle{Hipotesis 1}
  \begin{figure}[t]
    \caption{Data sin estandarizar}
    \includegraphics[width=5cm]{Figure_1.png}
  \end{figure}

\end{frame}

\begin{frame}
\frametitle{Hipotesis 1}
\begin{figure}[t]
  \caption{Data estandarizada y sin outliers}
  \includegraphics[width=6cm]{data_sin_outliers.jpeg}
\end{figure}
  \textbf{Puede parecer una distribucion Normal}

\end{frame}

\begin{frame}
\frametitle{Hipotesis 1}
\begin{figure}[t]
  \caption{Grafica Q-Q}
  \includegraphics[width=6cm]{grafiaq-q.png}
\end{figure}

  \textbf{Se aleja de la distribucion normal}
\end{frame}

\begin{frame}

  Se aplicó el test de \textit{Jarque-Bera}, para comprobar si la muestra
  presenta una \textbf{curtosis} y \textbf{asimetria} correspondientes
  a una ley normal.

  El estadistico de \textit{Jarque Bera} es asintoticamente un estimador de
  una \textit{Chi-Cuadrado} (${\chi_n ^ 2}$) y toma como hipotesis nula que los datos de la
  muestra siguen la ley normal

  \begin{alertblock}{Test de Jarque-Bera}
    \[\textbf{JB} = \frac{n}{6}(S^2 +\frac{1}{4}(K - 3)^3)\] Siendo \textit{n} los grados de libertad
  \end{alertblock}

  \begin{block}{Estimadores de momentos centrales}
    \begin{itemize}
        \item Tercer Momento Central
          \[S = \frac{\hat{\mu}_3}{\hat{\sigma}^3}\]

        \item  Cuarto Momento Central
         \[K = \frac{\hat{\mu}_4}{\hat{\sigma}^4}\]
    \end{itemize}
  \end{block}
\end{frame}

\begin{frame}
  Adicionalmente, se usara el test de \textit{Kolmogorov-Smirnov}, donde se plantea
  que la distribucion de ingresos en la poblacion de ciencia de datos
  no sigue la ley normal y se comparará con la funcion acumulada teoria
  de esta

  \[H_1: \textrm{La distribucion de ingresos \textbf{NO sigue} la ley normal}\]
  \[H_0: \textrm{La distribucion de ingresos \textbf{SIGUE} la ley normal}\]

\end{frame}

\begin{frame}
\frametitle{Conclusiones hipotesis 1}
  El test K-S y el de Jarque-Bera muestran los siguientes p-values.
  \begin{figure}[t]
        \includegraphics[width=12cm]{p-val-1.jpg}
    \end{figure}
    \begin{figure}[t]
          \includegraphics[width=12cm]{p-val-1.jpg}
      \end{figure}
    \end{frame}



\begin{frame}
\frametitle{Hipotesis 2}

\begin{figure}[h]
  \caption{Distribución de ingresos de ingenieros de
  software en la India}
  \includegraphics[width=6cm]{distribucion_ingresos_sw.jpeg}
\end{figure}

  se puede notar como existen \textit{2 grupos en la poblacion}
\end{frame}

\begin{frame}
  \frametitle{Distribución de los trabajadores en software}
  \alert{Aplicacion del test \textbf{Kolmogórov-Smirnov}}

  En este caso se va a comprar la funcion de distribucion acumulada observada
  con la de la distribucion teoria de una exponencial

\end{frame}

\begin{frame}
  \frametitle{Separando grupos aparentes}
  \includegraphics[width=8cm]{procesado.png}

\end{frame}

\begin{frame}
  \frametitle{Ajustando Curva}
  \includegraphics[width=8cm]{hip2/ajustando_curva.png}

\end{frame}

\begin{frame}
  \frametitle{Funciones acumuladas}
  \includegraphics[width=8cm]{hip2/acumuladas.png}

\end{frame}

\begin{frame}
  \frametitle{Grafico P-P}
  \includegraphics[width=8cm]{hip2/grafico_pp.png}
\end{frame}

\begin{frame}
  \frametitle{Conclusiones}
  De acuerdo al p-value obtenido no se puede rechazar la hipótesis nula
  \includegraphics[width=9cm]{hip2/p-val.jpg}
\end{frame}

\begin{frame}
  \frametitle{Hipotesis 3}

  Se aplicó el test de \textit{Kruskal-Wallis} con la finalidad de:

  \begin{itemize}
      \item Verificar si las muestras de Delhi y Bangalore provienen de poblacines
        distintas
        \item Comprobar si las 2 poblaciones difieren significativamente
  \end{itemize}

  Para esto se realizo un procedimiento similar al de la hipotesis anterior

  \begin{alertblock}{Consideracines del test}
    El test de Kruskal-Wallis es el sustituto no parametrico del
    \textit{"One way ANOVA"}, en el cual se necesita un factor independiente
  \end{alertblock}
\end{frame}

\begin{frame}
  \frametitle{Graficas de ambas ciudades}
  \begin{figure}[t]
      \includegraphics[width=12cm]{/hip8/ambas_graficas.png}
  \end{figure}

\end{frame}

\begin{frame}
  \frametitle{Graficas de Bangalore sin filtrar}
  \begin{figure}[t]
      \includegraphics[width=12cm]{/hip8/bangalore_sin_filtro.png}
  \end{figure}

\end{frame}

\begin{frame}
  \frametitle{Graficas de Delhi sin filtrar}
  \begin{figure}[t]
      \includegraphics[width=12cm]{/hip8/delhi_sin_filtro.png}
  \end{figure}
\end{frame}

\begin{frame}
  \frametitle{Resultados test}
  \begin{figure}[t]
      \includegraphics[width=12cm]{/hip8/resultados_test.jpg}
  \end{figure}
\end{frame}

\begin{frame}
  \frametitle{Conclusiones}
  \begin{enumerate}
   \item Aparecen 2 grupos en la poblacion de Bangalore
   \item Las distribuciones siguen la ley exponencial, solo
     varia su parametro de escalamiento
  \end{enumerate}
\end{frame}


\begin{frame}
  \frametitle{Comparacion de salarios DC y DI}
  \[{X_1}: \textrm{Salario de cientifico de datos} \rightarrow \overline{X_1} = 1061.79389312977, {\sigma_1}^2 = ?\]
  \[{X_2}: \textrm{Salario de ingeniero de datos} \rightarrow \overline{X_1} = 916.603773584, {\sigma_2}^2 = ?\]
\begin{figure}[t]
  \includegraphics[width=6cm]{cajas1.jpeg}
\end{figure}
\end{frame}

\begin{frame}
  \frametitle{Test de hipotesis}

 como se puede observar, las alturas de ambos diagras difieren significativamente, por lo que se consideran poblaciones
  con varianza diferentes

  Estadistico de prueba:
  \[
    \textrm{\textit{t}} = \frac{\overline{X}_1 - \overline{X}_2 - (\mu_1 - \mu_2)}{\sqrt{\frac{{S_1^2}}{n_1} + \frac{{S_2^2}}{n_2}}}
    \sim \textrm{\textit{t}}_{(\upsilon)}
  \]

  \[
    \upsilon = \frac{(\frac{S_1^2}{n_1} + \frac{S_2^2}{n_2})^2}{\frac{(\frac{S_1}{n_1})^2}{n_1 - 1} + \frac{(\frac{S_2}{n_2})^2}{n_2 - 1}}
  \]


  con un nivel de significancia $\alpha = 0.05$

\end{frame}

\begin{frame}
\frametitle{Conclusiones}
reemplazando datos:

  \textit{t} = 4,7424
  \newline
  $\upsilon = 149$
  \\
  \textbf{Region critica: }

  $t_{(149, 0.95)} = 1.655144$

  Se rechaza la hipotesis nula al ser el valor critico menor que el estadistico de prueba

  \textbf{Conclusion}\\
\end{frame}

\begin{frame}
\begin{figure}[t]
    \includegraphics[width=12cm]{Screenshot_20221028_234650.png}
\end{figure}

\end{frame}

\begin{frame}
  \frametitle{Comparacion de salarios publico y privado}
  \[{X_1}: \textrm{Salario publico} \rightarrow \overline{X_1} = 1110.3886, {\sigma_1}^2 = ?\]
  \[{X_2}: \textrm{Salario privado} \rightarrow \overline{X_privado} = 1020.8170, {\sigma_2}^2 = ?\]
\begin{figure}[t]
  \includegraphics[width=6cm]{cajas2.jpeg}
\end{figure}
\end{frame}

\begin{frame}
  \frametitle{Test de hipotesis}
\begin{figure}[t]
    \includegraphics[width=10cm]{Screenshot_20221028_235052.png}
\end{figure}
\end{frame}

\begin{frame}
  \frametitle{Distribucion de ingresos segun edad}
\begin{figure}[t]
    \includegraphics[width=10cm]{cuyate1.png}
\end{figure}
\end{frame}


\begin{frame}
  \frametitle{Distribucion de proporciones ingresos segun edad}
\begin{figure}[t]
    \includegraphics[width=10cm]{cuyate2.png}
\end{figure}
\end{frame}

\begin{frame}
  \frametitle{Comprobar que las personas que trabajan una cantidad
  mayor que la media perciben mejores ingresos
  }
  \begin{figure}[t]
      \includegraphics[width=10cm]{alvares1.png}
  \end{figure}
\end{frame}

\begin{frame}
  Para muestras grandes: \[ET = \sqrt{\frac{S_1 ^ 2}{n_1} + \frac{S_2 ^ 2}{n_2}}\]

  \begin{columns}
  \column{0.5\textwidth}
   \[H_0: \mu_1 = \mu_2\]
   \[H_1: \mu_1 > \mu_2\]

  \column{0.5\textwidth}

      \[\textit{Z} = \frac{\overline{X_1} - \overline{X_2}}{ET}\]
      \newline

  \end{columns}

  Dado un error maximo permitido $\alpha = 0.05$, con \textit{$H_1$} indicando una cola unilateral
  hacia la derecha $\rightarrow RC = {Z > 1.645}$

  \[Z_{cal} = \frac{45.4036}{38.8528} = 29.2267\]

  Como \textit{$Z_{cal} \in RC$} Se debe rechazar \textit{$H_0$} y concluir que
  aquellas personas que ganan mas de 50K trabaja en promedio mas que las personas
  que ganan menos de 50K

\end{frame}

\begin{frame}
  \frametitle{Comparar proporciones de trabajadores segun sexo}
  Sean $p_1 \textrm{ y } p_2$ la proporcion de trabajadores femeninos en \textit{Estados Unidos}
  y \textit{Mexico} respectivamente

  Siendo $n_1 = 14662,\textrm{ } x_1 = 4927 \textrm{ y } n_2 = 308, \textrm{ } x_2 = 69$ las cantidades totales
  y de poblacion femenina en ambos paises

  Dando como resultados:

  \[
    \hat{p} = \frac{4927 + 69}{14662 + 308} = 0.3245 \]
  \[
    \textrm{Error tipico: } \overline{p_1} - \overline{p_2}
  \]

  \[
    ET = \sqrt{\frac{\hat{p}(1 - \hat{p})}{n_1} + \frac{\hat{p}(1 - \hat{p})}{n_2}} = 0.02695
  \]

\end{frame}

\begin{frame}

  \begin{columns}
  \column{0.5\textwidth}
   \[H_0: p_1 = p_2\]
   \[H_1: p_1 > p_2\]

  \column{0.5\textwidth}

    Por \textbf{TCL}:
      \[\textit{Z} = \frac{\overline{p}_1 - \overline{p}_2}{ET} \sim N(0, 1)\]
      \newline

  \end{columns}

  Dado un error maximo permitido $\alpha = 0.05$, con \textit{$H_1$} indicando una cola unilateral
  hacia la derecha $\rightarrow RC = {Z > 1.645}$

  \[Z_{cal} = \frac{p_1 - p_2}{ET} = 4.15530\]

  Como \textit{$Z_{cal} \in RC$} Se debe rechazar \textit{$H_0$} y concluir que
  existe una mayor proporcion de mujeres trabajando en ciencia de datos en Estados
  Unidos que en Mexico; sin embargo, ese analisis no puede ser tan confiable debido
  a la diferencia del tamaño de las muestras

\end{frame}

\begin{frame}

  \frametitle{Otras conclusiones}

  \begin{itemize}
    \item Se logra visualizar la formacion de 2 grupos en la poblacion
      de Bangalore independientemente de la variable de analisis

    \item La distribucion teoria a la que mejor se aproximan los ingresos
      es la exponencial, la cual deriva de la distribución \textit{Gamma}

    \item Los test no parametricos son muy suceptibles a :
      \begin{enumerate}
        \item outliers
        \item Distribuciones con ligeras desviaciones de las teoricas
        \item Gran cantidad de data
      \end{enumerate}
  \end{itemize}


\end{frame}

\end{document}
