\begin{frame}
    \frametitle{Test de Wilcoxon}
    Si bien, el test de signos puede cumplir la misma funcion que el de
    \textbf{Wilcoxon}, este ultimo tiene mayor potencia al momento de
    detectar diferencia de medias.
  
\end{frame}

\begin{frame}
    \frametitle{Prueba de Wilcoxon}
    Se realiza una prueba de Wilcoxon para 2 muestras
    \textbf{relacionadas} 
    \includegraphics[width=1\textwidth]{cap/images/wilcoxon/repaso.jpg}
\end{frame}

\begin{frame}
    \frametitle{Prueba de Wilcoxon}
    \includegraphics[width=1\textwidth]{cap/images/wilcoxon/semestral.jpg}
\end{frame}

\begin{frame}

    \frametitle{Prueba de Wilcoxon}

    \[H_0: \textrm{Las distribuciones son semejantes}\]
    
    \[H_1: \textrm{Las distribuciones se encuentran desplazadas}\]

    Dado que el estadistico de prueba: \[ Z = 2.991 \] es mayor al 
    \[Z_crit = 1.645\] se rechaza la hipotesis nula, por lo que se no
    se puede afirmar que ambas muestras sean identicas.

    \textbf{El repaso no ayudó a mejorar los conocimientos}
\end{frame}

\begin{frame}
    \frametitle{Prueba de Wilcoxon}
    \includegraphics[width=1\textwidth]{cap/images/wilcoxon/resultados.jpg}
\end{frame}