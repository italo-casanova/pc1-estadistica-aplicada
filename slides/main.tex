\documentclass{beamer}
\usepackage[spanish]{babel}
\usepackage[utf8]{inputenc}

\usetheme{Madrid}
\usecolortheme{default}

%------------------------------------------------------------
%This block of code defines the information to appear in the
%Title page
\title[PC1 Estadistica Aplicada] %optional
{Aplicacion de tecnicas de estimacion y prueba de hipotesis}

\subtitle{Caso: Salario de profesionales}

\author % (optional)
{
  Alvarez \and Bautista \and Burga \and
  Casanova \and  Cuyate
}

\institute
{
  \inst{1}%
  Facultad de Ingenieria Industrial y de Sistemas\\
  Universidad Nacional de Ingenieria
}

\date
{ Octubre 2022}

% \logo{\includegraphics[height=1cm]{overleaf-logo}}

%End of title page configuration block
%------------------------------------------------------------

%------------------------------------------------------------
%The next block of commands puts the table of contents at the
%beginning of each section and highlights the current section:

\AtBeginSection[]
{
  \begin{frame}
    \frametitle{Table of Contents}
    \tableofcontents[currentsection]
  \end{frame}
}
%------------------------------------------------------------


\begin{document}

%The next statement creates the title page.
\frame{\titlepage}


%---------------------------------------------------------
%This block of code is for the table of contents after
%the title page
\begin{frame}
\frametitle{Table of Contents}
\tableofcontents
\end{frame}
%---------------------------------------------------------

\section{Problema}

\begin{frame}
\frametitle{Problematica}

\begin{itemize}
    \item Empiricamente se observa que en el mundo la precarizacion del trabajo
se acrecenta cada vez mas, de igual modo con la proporcion de personas
casadas y los salarios promedio de los jovenes
    \item Con motivo de generar informacion util para la prediccion de estas tendencias
socio-economicas se ha procedio a realizar un analisis estadistico sobre
9 hipotesis planteadas

\end{itemize}
\end{frame}



%---------------------------------------------------------
\section{Objetivos}

%---------------------------------------------------------
%Changing visivility of the text
\begin{frame}

\frametitle{Objetivos del trabajo}

\begin{alertblock}{General}
  Generar informacion relevante para la prediccion de tendencias
  socio-economicas en el mundo tomando como referencia datos
  que provienen mayoritariamente de Estados Unidos
\end{alertblock}
\end{frame}

\begin{frame}
\frametitle{Hipotesis especificas}

\begin{columns}
\column{0.5\textwidth}

\begin{itemize}
    \item La distribucion de ingresos de las personas jovenes sigue
      la ley normal
    \item En \alert{promedio} las personas con
      estudios universitarios terminados poseen una mejor distribucion
      de ingresos
    \item Los hombres de mediana edad que no se encuentran casados
      tienen una mejor distribucion de ingresos que los que si estan casados
\end{itemize}

\column{0.5\textwidth}

\begin{itemize}
    \item El promedio de ingresos de la poblacion mexicana es mayor
      que la peruana
    \item La varianza de ingresos de la poblacion mexicana es mayor
      que la peruana
    \item El \alert{promedio de ingresos} de las personas que trabajan
      una cantidad de horas superior a la mediana es mayor al promedio
      de ingreso de personas que laburan una cantidad menor de horas
      que la mediana
\end{itemize}
\end{columns}
\end{frame}

%---------------------------------------------------------

\begin{frame}
\frametitle{Hipotesis especificas}
\begin{itemize}
    \item Las personas de mediana edad poseen una mejor distribucion
      de ingreso que las personas jovenes
    \item \alert{por cambiar:} Los hombres poseen una mejor distribucion de ingresos
      que las mujeres
    \item Las peronas con le maximo grado academico tienen una distribucion
      distinta y con un promedio mayor a las demas categorias de educacion
\end{itemize}

\end{frame}

%---------------------------------------------------------


%---------------------------------------------------------
%Example of the \pause command
\begin{frame}
  floro del primer objetivo
\end{frame}
%---------------------------------------------------------

\section{Second section}

%---------------------------------------------------------
%Highlighting text
\begin{frame}
\frametitle{Sample frame title}

In this slide, some important text will be
\alert{highlighted} because it's important.
Please, don't abuse it.

\begin{block}{Remark}
Sample text
\end{block}

\begin{alertblock}{Important theorem}
Sample text in red box
\end{alertblock}

\begin{examples}
Sample text in green box. The title of the block is ``Examples".
\end{examples}
\end{frame}
%---------------------------------------------------------


%---------------------------------------------------------
%Two columns
\begin{frame}
\frametitle{Two-column slide}

\begin{columns}

\column{0.5\textwidth}
This is a text in first column.
$$E=mc^2$$
\begin{itemize}
\item First item
\item Second item
\end{itemize}

\column{0.5\textwidth}
This text will be in the second column
and on a second tought this is a nice looking
layout in some cases.
\end{columns}
\end{frame}
%---------------------------------------------------------

\end{document}
