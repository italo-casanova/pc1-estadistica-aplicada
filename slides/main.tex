\documentclass{beamer}
\usepackage[spanish]{babel}
\usepackage[utf8]{inputenc}

\usetheme{Madrid}
\usecolortheme{default}

%------------------------------------------------------------
%This block of code defines the information to appear in the
%Title page
\title[PC1 Estadistica Aplicada] %optional
{Aplicacion de tecnicas de estimacion y prueba de hipotesis}

\subtitle{Caso: Salario de profesionales}

\author % (optional)
{
  Alvarez \and Bautista \and Burga \and
  Casanova \and  Cuyate
}

\institute
{
  \inst{1}%
  Facultad de Ingenieria Industrial y de Sistemas\\
  Universidad Nacional de Ingenieria
}

\date
{ Octubre 2022}

% \logo{\includegraphics[height=1cm]{overleaf-logo}}

%End of title page configuration block
%------------------------------------------------------------

%------------------------------------------------------------
%The next block of commands puts the table of contents at the
%beginning of each section and highlights the current section:

\AtBeginSection[]
{
  \begin{frame}
    \frametitle{Table of Contents}
    \tableofcontents[currentsection]
  \end{frame}
}
%------------------------------------------------------------


\begin{document}

%The next statement creates the title page.
\frame{\titlepage}


%---------------------------------------------------------
%This block of code is for the table of contents after
%the title page
\begin{frame}
\frametitle{Table of Contents}
\tableofcontents
\end{frame}
%---------------------------------------------------------

\section{Problema}

\begin{frame}
\frametitle{Problematica}

floro sobre la problematica

\end{frame}



%---------------------------------------------------------
\section{Objetivos}

%---------------------------------------------------------
%Changing visivility of the text
\begin{frame}

\frametitle{Objetivos del trabajo}

\begin{alertblock}{General}
  % Comprobar por medio de herramientas de estadistica inferencial
  % si, en promedio factores como la educacion o el grado academico
  % influyen en el exito economico de las personas, asi como verifcar
  % si estas variables son causa de separcion de las poblaciones
  Generar informacion relevante para la prediccion de tendencias
  socio-economicas en el Peru comparando
\end{alertblock}
\end{frame}

\begin{frame}
\frametitle{Objetivos especificos}
\begin{columns}

\column{0.5\textwidth}
\begin{itemize}
    \item Comprobar si \alert{en promedio} las personas con
    estudios
    \item objeivo 2
    \item objeivo 3
    \item objeivo 4
    \item objeivo 5
\end{itemize}

\column{0.5\textwidth}

\begin{itemize}
    \item objeivo 6
    \item objeivo 7
    \item objeivo 8
    \item objeivo 9
\end{itemize}
\end{columns}

\end{frame}

%---------------------------------------------------------


%---------------------------------------------------------
%Example of the \pause command
\begin{frame}
In this slide \pause

the text will be partially visible \pause

And finally everything will be there
\end{frame}
%---------------------------------------------------------

\section{Second section}

%---------------------------------------------------------
%Highlighting text
\begin{frame}
\frametitle{Sample frame title}

In this slide, some important text will be
\alert{highlighted} because it's important.
Please, don't abuse it.

\begin{block}{Remark}
Sample text
\end{block}

\begin{alertblock}{Important theorem}
Sample text in red box
\end{alertblock}

\begin{examples}
Sample text in green box. The title of the block is ``Examples".
\end{examples}
\end{frame}
%---------------------------------------------------------


%---------------------------------------------------------
%Two columns
\begin{frame}
\frametitle{Two-column slide}

\begin{columns}

\column{0.5\textwidth}
This is a text in first column.
$$E=mc^2$$
\begin{itemize}
\item First item
\item Second item
\end{itemize}

\column{0.5\textwidth}
This text will be in the second column
and on a second tought this is a nice looking
layout in some cases.
\end{columns}
\end{frame}
%---------------------------------------------------------

\end{document}
